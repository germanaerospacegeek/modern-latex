\chapter{Installation}
\label{installation}

When you install \LaTeX{} on your computer, it comes packaged as a
\introduce{distribution} that contains:
\begin{enumerate}
\item \LaTeX, the program---the thing that turns markup into
    typeset documents.\footnote{Well, actually, multiple \LaTeX{} programs,
    but we're getting to that.}
\item A common set of \LaTeX{} \introduce{packages}.
    Packages are bundles of code that do all sorts of things,
    like provide new commands or change a document's style.
    We'll see a lot of them in action throughout this book.
\item Miscellaneous tools, like editors.
\end{enumerate}
Each major operating system has its own \LaTeX{} distribution:
\begin{description}
\item[Mac OS] has Mac\TeX. Grab it from \http{www.tug.org/mactex}
    and install it using the instructions there.

\item[Windows] has Mik\TeX.
    Install it from \https{miktex.org/download}.
    Mik\TeX{} has the helpful ability to automatically download
    additional packages as your documents use them for the first time.

\item[Linux and BSD] use \TeX{} Live.
    Like most software, it is provided through your
    \acronym{os}'s package manager.
    Linux distributions usually offer a \texttt{texlive-\allowbreak full}
    or \texttt{texlive-\allowbreak most} package that installs everything
    you need.\punckern\footnote{%
    If you would rather keep the install size down,
    Linux distributions usually break \TeX{} Live into multiple packages.
    Look for ones with names like
    \texttt{texlive-\allowbreak core}, \texttt{texlive-\allowbreak luatex}
    and \texttt{texlive-\allowbreak xetex}.
    As you work with \LaTeX, you may need less-common packages,
    which usually have names like \texttt{texlive-\allowbreak latexextra},
    \texttt{texlive-\allowbreak science}, and so on.
    Of course, all of this may vary from one Linux distribution to another.}
\end{description}

\section{Editors}

Since \LaTeX{} source files are regular text files,
you write them with the usual choices: Vim, Emacs,
Sublime, VS~Code, and so on.\punckern\footnote{If you've never used
any of these, try a few.
They're popular with programmers and other folks who shuffle text around
screens all day. Just don't use Notepad. Life is too short.}
There are also editors designed specifically for \LaTeX{},
which often come with their own built-in \acronym{pdf} viewer.
(You can find a fairly comprehensive list on the \LaTeX{} Wikibook,
in its installation chapter. See Appendix~\ref{resources}.)

\section{Online options}

If you don't want to install \LaTeX{} on your computer,
try online editors like Share\LaTeX{} or Overleaf.
This book doesn't focus on these web-based tools,
but the same basics apply.
Of course, you have less control over certain aspects,
like available fonts, the version of \LaTeX{} used, and so on.
